\documentclass[
  fontsize=12,
  english,
  paper=letter,
  headsepstyle=on,
  footinclude=off]
  {scrartcl}
\usepackage{scrpage2}
% \usepackage[footinclude=off]{typearea}
% \usepackage[letterpaper,margin=1in,driver=pdftex]{geometry}

\clearscrheadfoot                 % deletes header/footer
\pagestyle{scrheadings}           % use following definitions for header/footer
% definitions/configuration for the header
\lehead[]{\pagemark} % equal page, left (outer) position
\rohead[]{\pagemark}

% This lets us load OpenType fonts from the system without much fuss.
% It's already required by unicode-math, but I'm not using math here (and it's
% worth being redundant)!
\usepackage{fontspec}

% What can I say, I love the french
\frenchspacing

% microtype enables subtle typography improvements.
\usepackage{microtype}
\DisableLigatures{family = tt*}

% This keeps things like `` and --- working, even though I use “ and ” now...
\defaultfontfeatures{Ligatures=TeX}

\setmainfont{Linux Libertine O}
\setsansfont{Linux Biolinum O}
\setmonofont{Source Code Pro}
% \setmonofont{Inconsolata-g} % Skinny and easy on the eyes
% \setmathfont{TG Pagella Math}

% This is i18n overkill
% Also - folks seem to think polyglossia is better now (but support in LuaTeX is
% still a WIP in TeX Live 2013)
\usepackage{polyglossia}
\setdefaultlanguage[variant=american]{english}

% Disable morpheme-boundary-crossing ligatures
\usepackage[american]{selnolig}

% i18n of quotes - use for biblatex, but also potentially useful outside
% \usepackage{csquotes}

%% For bib if I end up using it.

% uniquelist fixes a bug in mis-interpreting APA section 6.12, fixed in v. 6.1
% of the apa style
% \usepackage[style=apa,backend=biber]{biblatex}

% More i18n overkill
% \DeclareLanguageMapping{american}{american-apa}

% \addbibresource{BIDS-app-sp2014.bib}

% Other setup:

\usepackage{underscore}
% Make it easier to print complicated tables
% \usepackage{longtable,siunitx,tabu}

% Better spacing and rules (lines, not laws)
% \usepackage{booktabs}

% \usepackage[small,compact]{titlesec}

\usepackage{enumitem}
\setlist{itemsep=0pt}

% \usepackage{minted}

% Note: with BibLaTeX, you should load hyperref after the biblatex package. And
% according to the WikiBook - it should be loaded close to last.
% pdfusetitle doesn't do much here (there's no TOC), but may as well keep it.
\usepackage[hidelinks,pdfusetitle]{hyperref}


% equal page, right position (inner)
\rehead[]{Dav Clark -- \textbf{Proposal} -- Data Science for Social Impact}
% odd   page, left  position (inner)
\lohead[]{Dav Clark -- \textbf{Proposal} -- Data Science for Social Impact}
% Describe the intended research project in more detail, including the
% anticipated impact of your proposed project on advancing scientific discovery
% in your field. (3 page maximum)

% I had been working on these as a separate application:
% 
%  - Could also involve Jon & Richmond Bay, or more generally Bay Area hub (Richmond + Oakland, etc)
%  - Proposal to BIDS as instructional development grant on picking up ill-fitting
%    project from DSSG.io and existing local
%  - $20k as a starting budget for a GSR to work on this
%  - Anno could support a GSR?
%  - Mention Bay Area hub, but call out Richmond Bay.

\begin{document}

% \title{Data Science for Social Impact}
% \author{Dav Clark} % must match BearFacts!
% \maketitle
\manualmark
\pagestyle{scrheadings}
\markleft{}

The D-Lab and BIDS both have the same fundamental problem: how do we get
collaboration right? We are in an era where science could markedly decrease
human suffering and species loss on a global scale. I fear, however, that the
practice of science is heavily proscribed by seemingly insurmountable technical
challenges. In my experience, these challenges are often not even
\emph{difficult} to a relevant domain expert.
% Amp up pathbreaking / cutting-edge research. Wings of the T - but make that
% more concrete, perhaps with examples. Point out in particular research we
% couldn't do before.
I suspect I have strong alignment with the vision of the BIDS team that it is
\emph{unacceptable} for science to remain limited by our inability to share our
technical capacities.

% What is the "great project"?

Below, I'll discuss how I've personally encoutered the importance of
collaboration in my own research. Then, I'll describe how I'd like to help BIDS
develop effective collaborations, both internally and externally.

\subsection*{A case study: Climate change education}
% Maybe mention the flip w/ liberals and nuclear power?

Climate!
Oroeco
Campus partners - Kevin / GIF, Karthik?, Zeke / BEST, Ranney, etc., Clayton Critcher?
Identified problems with physical / behavioral systems (cf. Sol, Ted Miguel)
Quote via Steve Lewendowski (IPCC guy - problem is behavioral, not physical), more likely Elke Weber
Garrison, BECC, etc.

The urgency and importance of meeting the challenge of scientific collaboration
became particularly apparent in my thesis work on climate change education.
There, models of increasing technical and scientific sophistication are
continually increasing our certainty and understanding of the very real dangers
facing our world. It's no secret, however, that meaningful policy change has
been extraordinarily difficult to bring about. Indeed, only in the most recent
IPCC report (the fifth so far) was a social scientist (Elke Weber) invited to
contribute to recommendations for mitigating the likely dire consequences of
global climate change.

% http://theconversation.com/the-truth-is-out-there-so-how-do-you-debunk-a-myth-22641?utm_source=feedburner&utm_medium=feed&utm_campaign=Feed%3A+conversationedu+(The+Conversation)
John Cook at Skeptical Science and the University of Queensland might be
considered a data science pioneer in the climate and behavior field. He runs
what is likely the highest profile “climate truth” website, and had a successful
campaign largely via Twitter and Facebook last year to fight misinformation
about scientific consensus with this verified fact: over 97\% of publishing
climate researchers accept the reality of human-caused climate change.

I am not a pioneer. I am chief scientist at Oroeco, where we take a
consumer-spending approach to reducing carbon footrint.  Our general approach
has been tried at least twice before (at WattzOn and Efficiency 2.0) and those
businesses have failed.\footnote{Technically, they pivoted. A quick search will
    reveal that both brands are still around.} However, we have MacArthur
“genius” Saul Griffith -- a founder of WattzOn -- as an advisor and he believes
it's worth another shot, especially given advances in working through social
media and mobile interactivity. Likewise, Zeke Hausfather -- a founding member
of Efficiency 2.0 who is now working with Richard Mueller at the Berkeley Earth
Institute -- is working with us on implementing effective tips and goals for our
users. I hope \emph{you} agree that it's worth a try.

But the reality is that almost no one in the climate and behavior field has the
“hacking skills” needed to do truly first-rate data science in the field of
climate and behavior. To any likely reader of this application, the research
done by John Cook, Saul Griffith, and Zeke Hausfather is pretty simple-minded.
Climate scientists certainly have access to incredible hacking skills for
solving massive parallel systems of dynamic equations, but they don't
have the combination of social-behavioral science and facility with
web technologies and social media required to do a the job we need to be doing.

Species are going extinct, % cite?
and humans are already suffering as a consequence of global climate change. %Hsiang
This will continue to happen, but it is within our power to make it a little
more or a little less.

By myself, I don't have the skills or bandwidth to properly design, evaluate, and rapidly
iterate on the conceptual and behavioral change interventions that would
mitigate climate change and its impacts. I certainly don't have the documented
billion dollars that have been spent on anti-science misinformation.

I need collaborators, and I'd love to more formally collaborate with you at BIDS.

\subsection*{So what \emph{can} I do?}

% Stuff from talking with Seb
% MGH, multimodal imaging
% Background in computational linguistics
% Crowdsourcing, fluency with latent variable models
% Engagement with real world problems
% DSSG

Before I came to Berkeley, I developed novel analysis technique using
inter-subject correlation in fMRI signals to observe the neural correlates of
ecological learning.\footnote{Specifically, viewing an episode of \emph{Curb
Your Enthusiasm}} I developed suitable algorithms for the taxonomic
categorization of web pages for online advertising. Then, I started working with
Rich Ivry on the neural basis of motor skill learning.

All of this contributed in some respect to my thesis research: evaluating a
number of climate education interventions using crowdsourcing on Amazon
Mechanical Turk. I implemented a solution in JavaScript to prevent Turkers from
enrolling in more than one of my experiments (a common problem). %cite?  
I also implemented a simple geo-IP based tool to determine whether individuals
were correctly reporting their location (two participants had in fact managed to
bypass Amazon restrictions on U.S.-only participants). Lastly, I implemented a
hybrid natural langauge processing solution / google web services solution to
check for individuals copying textual answers from the web instead of honestly
answering the question (turns out several people did this, even though there was
no consequence of answering poorly, and I told them I'd provide the correct
answer later).

If you've ever done this kind of work, it will come as no surprise that I also
dealt with numerous unicode issues.

With my team at Oroeco, I've set up an analytics environment using Vagrant such
that, even though he has no idea what he's actually doing, our CEO can spin up a
Linux VM, sync our live database to a local PostgreSQL server, and look at
up-to-the-minute visualizations in IPython notebooks that we share via GitHub.

\subsection*{Collaboration in the D-Lab and BIDS}

I've been working at the D-Lab on campus-wide collaboration since last fall. In
fact, I've fascilitated arguably the first large BIDS offering in conjunction with Software Carpentry.
Meanwhile, I'm working with Rachel Slaybough \& Katy Huff on planning improved
assessment for their Software Carpentry / Hacker Within initiative, as well as
the D-Lab's own initiative on equity and inclusion. Note also training mission
with BITSS.


\subsubsection*{Methods Modules}

ENABLING PATHBREAKING / CUTTING-EDGE RESEARCH.

I am currently engaged on a number of fronts with the creation of modular
content that can help individuals correctly and efficiently select and apply
computational techniques to their scientific question.

\begin{enumerate}
    \item Engaging in the dialogue between science and power (i.e., money,
        religion)

    \item Working with the newly funded Psychological Data Science center (TODO
        - get correct title) with Tom Griffiths.

        List types of data and devices / methodologies.

    \item BITSS Course

    \item Twitter data

    \item Bringing together a platform for collaboration - Xmobile, Pablo,
        Oroeco, Chris Jones / CoolClimate

    \item Virtual telepresence with Marshall
\end{enumerate}

\subsubsection*{Curated Data Sets}

I'm struck that the scope of what we're discussing is a bit intractable at
present. I have three focused “data missions” currently, the second of which is
shared with Jon:

\begin{enumerate}
    \item Organize free flow of data between both physical and behavioral
        research surrounding climate change.

        The UC Berkeley campus (+ LBNL \& friends) has an amazing wealth of such
        researchers, and they all whittle away in silos that render their work
        fairly impotent. An index of climate / ecology data from UC Berkeley
        provides a challenging project, and also seems like a good point of
        connection with Karthik (the engine behind rOpenSci -- which I was
        suggesting is the best first approximation I know of an “open data”
        index).

    \item Organize a dataset surrounding the Richmond Bay campus.

        This is a major initiative that will have huge impacts on the city of
        Richmond.  On the flipside, the city of Richmond is the largest city
        with a green mayor (a signal of strong liberalism), and there are
        existing examples of high school students working productively with
        municipal datasets.

    \item Coordinate with the Data Science for Social Good fellows program
        (\url{http://dssg.io}).

        I've been working on ramping up a bay area hub to coordinate with DSSG,
        picking up community partners with analysis problems they can't
        currently solve, and finding “data scientists” who can help at Berkeley
        and Stanford (and later, elsewhere). I've got a nascent site up with
        Lester Mackey \& Andreas Stuhlmüller at Stanford:
        http://statsforchange.org

    \item Enabling collaborative evaluation with D-Lab / Software Carpentry style
        technical training workshops.

        The D-Lab is both hosting SW Carpentry workshops, as well as
        coordinating with Rachel Slaybaugh (professor in Nuke Eng.) and Katy
        Huff on assessment with a focus on equity and inclusion. In parallel,
        we've been ramping up our own efforts in this domain. Open data would
        allow for more of a “citizen science” approach that would be coherent
        with the Software Carpentry approach to grass-roots instruction.

\end{enumerate}

\end{document}
