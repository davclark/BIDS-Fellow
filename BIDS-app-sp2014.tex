\documentclass[12pt]{article}

% This lets us load OpenType fonts from the system without any fuss. 
% It's already required by unicode-math, but I include it for reference.
\usepackage{fontspec}

% What can I say, I love the french
\frenchspacing

% microtype enables subtle typography improvements.
\usepackage{microtype}
\DisableLigatures{family = tt*}

% This keeps things like `` and --- working
\defaultfontfeatures{Ligatures=TeX}

\setmainfont{Linux Libertine O}
\setsansfont{Linux Biolinum O}
\setmonofont{Inconsolata} % Skinny and easy on the eyes
% \setmathfont{TG Pagella Math}

\usepackage[letterpaper,margin=1in,driver=pdftex]{geometry}

% Bibliography stuff:
% This is 1i18n overkill, including i18n of quotes
% Also - folks seem to think polyglossia is better now (but it only works in
% XeTeX, not LuaTeX in TeX Live 2012)
% \usepackage[american]{babel}
% \usepackage{csquotes}

% uniquelist fixes a bug in mis-interpreting APA section 6.12, fixed in v. 6.1
% of the apa style
% \usepackage[style=apa,uniquelist=minyear,backend=biber]{biblatex}

% More i18n overkill
% \DeclareLanguageMapping{american}{american-apa}

% \addbibresource{BIDS-app-sp2014.bib}

% Other setup:

% \usepackage{underscore}
% Make it easier to print complicated tables
% \usepackage{longtable,siunitx,tabu}

% Better spacing and rules (lines, not laws)
% \usepackage{booktabs}

% \usepackage{enumitem}
% \setlist{itemsep=0pt}

% \usepackage{minted}

% Note: with BibLaTeX, you should load hyperref after the biblatex package. And
% according to the WikiBook - it should be loaded close to last.
% \usepackage[pdfborder={0 0 0},pdfusetitle]{hyperref}

\begin{document}
The D-Lab and BIDS both have the same fundamental problem: how do we get
collaboration right? Below, I'll discuss how I've personally encoutered the
importance of collaboration in my own research. Then, I'll describe how I'd like
to help BIDS develop effective collaborations, both internally and externally.

\subsection*{A case study: Climate change education}

The urgency and importance of meeting the challenge of scientific collaboration
became particularly apparent in my thesis work on climate change education.
There, models of increasing technical and scientific sophistication are
continually increasing our certainty and understanding of the very real dangers
facing our world. It's no secret, however, that meaningful policy change has
been extraordinarily difficult to bring about. Indeed, only in the most recent
IPCC report (the fifth so far) was a social scientist (Elke Weber) invited to
contribute to recommendations for mitigating the likely dire consequences of
global climate change.

% http://theconversation.com/the-truth-is-out-there-so-how-do-you-debunk-a-myth-22641?utm_source=feedburner&utm_medium=feed&utm_campaign=Feed%3A+conversationedu+(The+Conversation)
John Cook at Skeptical Science and the University of Queensland might be
considered a data science pioneer in the climate and behavior field. He runs
what is likely the highest profile “climate truth” website, and had a successful
campaign largely via Twitter and Facebook last year to fight misinformation
about scientific consensus with this verified fact: over 97\% of publishing
climate researchers accept the reality of human-caused climate change. 

I am not a pioneer. I am chief scientist at Oroeco, where we take a
consumer-spending approach to reducing carbon footrint.  Our general approach
has been tried at least twice before (at WattzOn and Efficiency 2.0) and those
businesses have failed.\footnote{Technically, they pivoted. A quick search will
    reveal that both brands are still around.} However, we have MacArthur
“genius” Saul Griffith -- the founder of WattzOn -- as an advisor and he
believes it's worth another shot, especially given advances in working through
social media and mobile interactivity. Likewise, a founding member of Efficiency
2.0 who is now working with Richard Mueller at the Berkeley Earth Institute --
Zeke Hausfather -- is working with us on implementing effective tips and goals
for our users. I hope you agree with us that this it's worth a try.

But the reality is that almost no one in the climate and behavior field has the
“hacking skills” needed to do truly first-rate data science in the field of
climate and behavior. To any likely reader of this application, the research
done by John Cook, Saul Griffith, and Zeke Hausfather is pretty simple-minded.
Climate scientists certainly have access to incredible hacking skills for
solving massive parallel systems of dynamic equations, but they don't
have the combination of social-behavioral science and facility with
web technologies and social media required to do a the job we need to be doing.

Species are going extinct, % cite?
and humans are already suffering as a consequence of global climate change. %Hsiang
This will continue to happen, but it is within our power to make it a little
more or a little less.

I don't have the skills or bandwidth to properly design, evaluate, and rapidly
iterate on the conceptual and behavioral change interventions that would
mitigate climate change and its impacts. I certainly don't have the documented
billion dollars that have been spent on anti-science misinformation. 

I need collaborators. 

\subsection*{So what \emph{can} I do?}

Novel analysis technique using inter-subject correlation in fMRI signals to
observe the neural correlates of ecological learning.\footnote{Specifically,
    viewing an episode of \emph{Curb Your Enthusiasm}}

For my thesis research, I evaluated a number of interventions using
crowdsourcing on Amazon Mechanical Turk. I implemented a solution in JavaScript
to prevent Turkers from enrolling in more than one of my experiments (a common
problem). %cite?
I also implemented a simple geo-IP based tool to determine whether individuals were
correctly reporting their location (two participants had in fact managed to
bypass Amazon restrictions on U.S.-only participants). Lastly, I implemented a
hybrid natural langauge processing solution / google web services solution to
check for individuals copying textual answers from the web instead of honestly
answering the question (turns out several people did this, even though there was
no consequence of answering poorly, and I told them I'd provide the correct
answer later).
Oh, and I also dealt with unicode issues.

With my team at Oroeco, I've set up an analytics environment using Vagrant such
that, even though he has no idea what he's actually doing, our CEO can spin up a
Linux VM, sync our live database to a local PostgreSQL server, and look at
up-to-the-minute visualizations in IPython notebooks that we share via GitHub.

\subsection*{Collaboration in the D-Lab and BIDS}

I've been working at the D-Lab on this problem since 
\end{document}
