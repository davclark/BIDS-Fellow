\documentclass[12pt]{article}

% This lets us load OpenType fonts from the system without any fuss. 
% It's already required by unicode-math, but I include it for reference.
\usepackage{fontspec}

% What can I say, I love the french
\frenchspacing

% microtype enables subtle typography improvements.
\usepackage{microtype}
\DisableLigatures{family = tt*}

% This keeps things like `` and --- working
\defaultfontfeatures{Ligatures=TeX}

\setmainfont{Linux Libertine O}
\setsansfont{Linux Biolinum O}
\setmonofont{Inconsolata} % Skinny and easy on the eyes
% \setmathfont{TG Pagella Math}

% geometry driver auto detection may not work for XeLaTeX below
% \usepackage[letterpaper,includehead,margin=1in,headheight=15pt,showframe]{geometry}
\usepackage[letterpaper,margin=1in]{geometry}

% Bibliography stuff:
% This is 1i18n overkill, including i18n of quotes
% Also - folks seem to think polyglossia is better now (but it only works in
% XeTeX, not LuaTeX in TeX Live 2012)
% \usepackage[american]{babel}
% \usepackage{csquotes}

% uniquelist fixes a bug in mis-interpreting APA section 6.12, fixed in v. 6.1
% of the apa style
% \usepackage[style=apa,uniquelist=minyear,backend=biber]{biblatex}

% More i18n overkill
% \DeclareLanguageMapping{american}{american-apa}

% \addbibresource{BIDS-app-sp2014.bib}

% Other setup:

% \usepackage{underscore}
% Make it easier to print complicated tables
% \usepackage{longtable,siunitx,tabu}

% Better spacing and rules (lines, not laws)
% \usepackage{booktabs}

% \usepackage{enumitem}
% \setlist{itemsep=0pt}

% \usepackage{minted}

% Note: with BibLaTeX, you should load hyperref after the biblatex package. And
% according to the WikiBook - it should be loaded close to last.
% \usepackage[pdfborder={0 0 0},pdfusetitle]{hyperref}

\begin{document}
The D-Lab and BIDS both have the same fundamental problem: how do we get
collaboration right? Below, I'll discuss how I've personally encoutered the
importance of collaboration in my own research. Then, I'll describe how I'd like
to help BIDS develop effective collaborations, both internally and externally.

\subsection*{A case study: Climate change education}

The urgency and importance of meeting this challenge became
particularly apparent in my thesis work on climate change education. There,
models of increasing technical and scientific sophistication are continually
increasing our certainty and understanding of the very real dangers facing our
world. It's no secret, however, that meaningful policy change has been
extraordinarily difficult to bring about. Indeed, only in the most recent IPCC
report (the fifth so far) was a social scientist (Elke Weber) invited to
contribute to recommendations for mitigating the likely dire consequences of
global climate change.

John Cook at Skeptical Science might be considered a data science pioneer in the
climate and behavior field. He p. I am not a pioneer. I am chief scientist at
Oroeco, where we take a consumer-spending approach to reducing carbon footrint.
Our general approach has been tried at least twice before and those businesses
have failed.\footnote{Technically, they pivoted. A quick search will reveal that
both brands are still around.} However, we have the founder of the first
such company, Saul Griffiths (a MacArthur award winner, and founder of WattzOn)
as an advisor.  Likewise, a founding member of Efficiency 2.0 who is now working
with Richard Mueller at the Berkeley Earth Institute, Zeke Hausfather [sp?], is
working with us on implementing effective tips and goals for our users. Both of
these folks believe that given the advances in social media and behavioral
change technology that we might have a change. I hope you agree that it's worth
a try.

But the reality is that almost no one in the climate and behavior field has the
“hacking skills” needed to do truly first-rate data science in the field of
climate and behavior. Climate scientists certainly have access to incredible
hacking skills for solving massive parallel systems of dynamic equations, but
they don't necessarily have 

\subsection*{Collaboration in the D-Lab and BIDS}

I've been working at the D-Lab on this problem since 
\end{document}
