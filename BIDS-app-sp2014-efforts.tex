% This is an introductory call with multiple goals:

% a. To understand the wide landscape of data science efforts at Berkeley, widely
% defined to include LBNL, MSRI, all campus departments and research
% centers/institutes, including the humanities, sciences, engineering, and
% professional schools;

% b. To help frame and grow the mission of BIDS; and

% c. To expand our funding base and partner to reach new funding sources.

% Your response to the “call for participation” should be a 2-3 page document
% that describes your efforts in data science, including your challenges and
% opportunities. We are not asking for a budget or a formal proposal. Instead,
% please provide an overview of the scientific goals and vision and how working
% within a larger data science context will further that vision. Our intent is
% two-fold: To build a community of data science, and help us steer and evolve
% the mission of BIDS; and to help attract funding from other sources, including
% new grants, support from private donors, and corporate sponsorships.

\documentclass[
  fontsize=12,
  english,
  paper=letter,
  headsepstyle=on,
  footinclude=off]
  {scrartcl}
\usepackage{scrpage2}
% \usepackage[footinclude=off]{typearea}
% \usepackage[letterpaper,margin=1in,driver=pdftex]{geometry}

\clearscrheadfoot                 % deletes header/footer
\pagestyle{scrheadings}           % use following definitions for header/footer
% definitions/configuration for the header
\lehead[]{\pagemark} % equal page, left (outer) position
\rohead[]{\pagemark}

% This lets us load OpenType fonts from the system without much fuss.
% It's already required by unicode-math, but I'm not using math here (and it's
% worth being redundant)!
\usepackage{fontspec}

% What can I say, I love the french
\frenchspacing

% microtype enables subtle typography improvements.
\usepackage{microtype}
\DisableLigatures{family = tt*}

% This keeps things like `` and --- working, even though I use “ and ” now...
\defaultfontfeatures{Ligatures=TeX}

\setmainfont{Linux Libertine O}
\setsansfont{Linux Biolinum O}
\setmonofont{Source Code Pro}
% \setmonofont{Inconsolata-g} % Skinny and easy on the eyes
% \setmathfont{TG Pagella Math}

% This is i18n overkill
% Also - folks seem to think polyglossia is better now (but support in LuaTeX is
% still a WIP in TeX Live 2013)
\usepackage{polyglossia}
\setdefaultlanguage[variant=american]{english}

% Disable morpheme-boundary-crossing ligatures
\usepackage[american]{selnolig}

% i18n of quotes - use for biblatex, but also potentially useful outside
% \usepackage{csquotes}

%% For bib if I end up using it.

% uniquelist fixes a bug in mis-interpreting APA section 6.12, fixed in v. 6.1
% of the apa style
% \usepackage[style=apa,backend=biber]{biblatex}

% More i18n overkill
% \DeclareLanguageMapping{american}{american-apa}

% \addbibresource{BIDS-app-sp2014.bib}

% Other setup:

\usepackage{underscore}
% Make it easier to print complicated tables
% \usepackage{longtable,siunitx,tabu}

% Better spacing and rules (lines, not laws)
% \usepackage{booktabs}

% \usepackage[small,compact]{titlesec}

\usepackage{enumitem}
\setlist{itemsep=0pt}

% \usepackage{minted}

% Note: with BibLaTeX, you should load hyperref after the biblatex package. And
% according to the WikiBook - it should be loaded close to last.
% pdfusetitle doesn't do much here (there's no TOC), but may as well keep it.
\usepackage[hidelinks,pdfusetitle]{hyperref}


% definitions/configuration for the header
\rehead[]{Dav Clark -- \textbf{Proposal} -- Data Science for Social Impact}
\lohead[]{Dav Clark -- \textbf{Proposal} -- Data Science for Social Impact}
\lehead[]{\pagemark}
\rohead[]{\pagemark}

\begin{document}
% that describes your efforts in data science, including your challenges and
% opportunities. No budget or a formal proposal.

% Instead, please provide an overview of the scientific goals and vision

The D-Lab and BIDS both have the same fundamental problem: \emph{how do we get
    collaboration right?} The topic of my PhD thesis was climate change, and
during the course of my study I became fully aware that we are in an era where
science could markedly decrease human suffering and species loss on a global
scale. Unfortunately, the practice of science is often heavily proscribed by
seemingly insurmountable technical challenges and barriers to collaboration. In
my experience, many “insurmountable challenges” are not even \emph{difficult} to
a relevant domain expert. In decision making terms, there is “money on the
table” in the form of the vast expertise embodied by researchers at UC Berkeley.
I propose to work with BIDS to identify low-hanging fruit for collaboration that
can push the frontiers of our ability to understand and engineer solutions for
our world. I am well suited to such an endeavor given my particular areas of
deep and broad expertise combined with my passion for impactful science.

\subsection*{Data science for social impact}

If one had to pick “the” conference on data science, it would likely be the ACM KDD
(Knowledge Discovery and Data Mining) conference. The theme for KDD this year is
“Data Mining for Social Good” -- signaling serious recognition of the relevance
of social impact. In my research proposal, I'll go into more detail about my
particular passion for collaborative science that's been honed by my work on
climate change. In short, I know, and I think “the field” knows that social
impact is important.

\subsection*{How I became a data scientist}

% Stuff from talking with Seb
% MGH, multimodal imaging
% Background in computational linguistics
% Crowdsourcing, fluency with latent variable models
% Engagement with real world problems
% DSSG
% Start-Up Chile

Before I came to Berkeley, I had worked on the integration of high temporal
frequency MEG data with (relatively) high spatial frequency fMRI data at
Massachusetts General Hospital. Later, I developed novel analysis technique
using inter-subject correlation in fMRI signals to observe the neural correlates
of ecological learning at NYU and the Weizman Institute.\footnote{Specifically,
    viewing an episode of \emph{Curb Your Enthusiasm}} During a one-year job I
took while applying to UC Berkeley, I developed suitable algorithms for the
taxonomic categorization of web pages for online advertising.

All of this contributed in some respect to my thesis research: evaluating a
number of climate education interventions using crowdsourcing on Amazon
Mechanical Turk. I implemented a solution in JavaScript to prevent Turkers from
enrolling in more than one of my experiments (a common problem). %cite?  
I also implemented a simple geo-IP based tool to determine whether individuals
were correctly reporting their location (two participants had in fact managed to
bypass Amazon restrictions on U.S.-only participants). Lastly, I implemented a
hybrid of natural langauge processing with google web services solution to
check for individuals copying textual answers from the web instead of honestly
answering the question (turns out several people did this, even though there was
no consequence of answering poorly, and I told them I'd provide the correct
answer later).

If you've ever done this kind of work, it will come as no surprise that I also
dealt with numerous Unicode issues.

With my team at Oroeco, I've set up an analytics environment using Vagrant such
that, even though he has no idea what he's actually doing, our CEO can spin up a
Linux VM, sync our live database to a local PostgreSQL server, and look at
up-to-the-minute visualizations in IPython notebooks that we share via GitHub.

% how working within a larger data science context will further that vision.

\subsection*{Generalizing collaboration in BIDS and the D-Lab}

I've been working at the D-Lab on campus-wide collaboration since last fall. If
you refer to the “impacts” document in my application, you'll see a list of
things that I'm already doing to share general solutions to improve sceintific
collaboration. In fact, I've fascilitated arguably the first large public
offering from BIDS in conjunction with Software Carpentry. My primary approach
so far, consistent with the D-Lab, has been educational. I'm working with Rachel
Slaybough \& Katy Huff in nuclear engineering and our own equity and inclusion
team in the D-Lab on planning improved technical education and assessment. I'm
also organizing a track on computing skills for reproducibility at the upcoming
BITSS/ICPSR summer institute. Below are some examples of ways I could continue
that work.

\subsubsection*{Methods Modules}

Engineering proceeds in part by standardization of a level of complexity, so
that innovation can occur at a higher level. Currently methods education and
implementation are done in a relatively \emph{ad hoc} fashion. One approach
(described more fully in the “impacts” document) is to develop consistent
computing environments so that everyone on campus can be “on the same page” in
terms of sharing software and code. Philip Stark has generated some excitement
for this approach, in addition to individuals across campus on both the staff
and academic side.

A more interesting and ambitious project came out of my discussion with Rayid
Ghani and Lester Mackey about our “Data Science for Social Good” (DSSG) efforts.
In short, while DSSG was able to recruit technically talented graduate and
undergraduate students for their program last year, many of these students had
never been exposed to machine learning methods. Thus, a disproportionate amount
of time was spent navigating and “figuring out” specific data mining methods.
Our agreed-upon goal is to identify methods that have been broadly successful,
and provide an index that helps program participants (and ultimately, any
researcher) identify appropriate methods and readily apply them to their
question or problem domain. Meanwhile, Puneet Kishor (Science and Data Policy,
Creative Commons) has asked me to fascilitate collaboration with him and Peter
Murray-Rust (U of Cambridge) and Ross Mounce (U of Bath) on the development of a
text data mining curriculum.

Along similar lines, Tom Griffiths in psychology is in dialogue with us at the
D-Lab regarding his recently awarded grant for the creation of a Center for
Data-Intensive Psychological Science calls for enabling researcher access to
largely untapped resources, like image data on Flickr.  The flip side of this
demand is expressed in Xmobile -- where a certain level of capacity has been
built up to collect interesting data from mobile devices, but there's no clear
sense of how to scale the program.

I hope these examples suffice to illustrate the large need for well-organized,
accessible materials for implementing modern methods of data collection and
analysis. BIDS seems like the logical place to center such efforts on campus.

\subsubsection*{Curated Data Sets}

Above, I mentioned the effort involved in accessing data from the CoolClimate
project. I'm working with the Info-Energy Nexus to help that group collect and
curate data API libraries and locally cached datasets. Here in the D-Lab, Jon
Stiles and I are particularly interested in a data collection project
surrounding the impacts of the Richmond Bay campus on the surrounding
neighborhoods (and working to make those impacts positive!). Jon provides his
extensive experience as a data archivist, and Puneet Keshor at Creative Commons
has offered to lend his resources in navigating legal/copyright issues.

I'm also playing a supporting role in a data collection effort for Twitter,
coordinating with members of the D-Lab, the AMP lab, Civil Engineering, and
campus IT. This approach is highlighting some of the significant differences in
both conceptualization and technical implementation of data collection and
analysis.  It is clear that a campus-wide approach to dealing with continuous
streams of social media data is of high value.

The D-Lab is both hosting SW Carpentry workshops, as well as
coordinating with Rachel Slaybaugh (professor in Nuke Eng.) and Katy
Huff on assessment with a focus on equity and inclusion. In parallel,
we've been ramping up our own efforts in this domain. Open data would
allow for more of a “citizen science” approach that would be coherent
with the Software Carpentry approach to grass-roots instruction.

% Our intent is two-fold: To build a community of data science, and help us
% steer and evolve the mission of BIDS; and to help attract funding from other
% sources, including new grants, support from private donors, and corporate
% sponsorships.

\subsubsection*{A DSSG-style program at UC Berkeley}

It should be clear that the above projects would lead naturally to supporting a
Data Science for Social Good program here at Cal (perhaps in partnership with
the group at U Chicago). This would be a large project (at U Chicago, it was on
the order of \$1M), attracting some of the top “data science” talent in the
world to our campus for a summer. As Matt Gee from U Chicago's Urban Data Canter
has been discussing with companies, social impact can be an important factor in
recruitment, in addition to being appealing to funders.

A lot of great science can come out of taking seriously the question of social
impact. I have a particular focus on climate change, and I'm fortunate to have
an ally in persuing all three of the above categories of program with the
Information-Energy Nexus group. The opportunities for social impact are quite
varied and, frankly, incredibly exciting. Moreover, even the process of
transforming “basic” science to be more transparent and reliable is likely to
yeild understanding that can contribute to the wellbeing of our world. In all of
these cases, I expect that formerly untapped expertise will make things easy
that once seemed impossible. I look forward to exploring these opportunities
with BIDS.

\end{document}
