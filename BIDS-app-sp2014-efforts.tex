% This is an introductory call with multiple goals:

% a. To understand the wide landscape of data science efforts at Berkeley, widely
% defined to include LBNL, MSRI, all campus departments and research
% centers/institutes, including the humanities, sciences, engineering, and
% professional schools;

% b. To help frame and grow the mission of BIDS; and

% c. To expand our funding base and partner to reach new funding sources.

% Your response to the “call for participation” should be a 2-3 page document
% that describes your efforts in data science, including your challenges and
% opportunities. We are not asking for a budget or a formal proposal. Instead,
% please provide an overview of the scientific goals and vision and how working
% within a larger data science context will further that vision. Our intent is
% two-fold: To build a community of data science, and help us steer and evolve
% the mission of BIDS; and to help attract funding from other sources, including
% new grants, support from private donors, and corporate sponsorships.

\documentclass[
  fontsize=12,
  english,
  paper=letter,
  headsepline]
  {scrartcl}
\usepackage{scrpage2}
% \usepackage[footinclude=off]{typearea}
\usepackage[letterpaper,margin=1in,driver=pdftex]{geometry}

% gets rid of page numbering at bottom
\clearscrheadfoot
\pagestyle{scrheadings}
% Then define custom \rehead[]{Stuff} and \lohead[]{Stuff} in the document for
% the left side, and \lehead + \rohead for the right (prob. \pagemark or
% nothing)

% This lets us load OpenType fonts from the system without much fuss.
% It's already required by unicode-math, but I'm not using math here (and it's
% worth being redundant)!
\usepackage{fontspec}

% What can I say, I love the french
\frenchspacing

% microtype enables subtle typography improvements.
\usepackage{microtype}
\DisableLigatures{family = tt*}

% This keeps things like `` and --- working, even though I use “ and ” now...
\defaultfontfeatures{Ligatures=TeX}

\setmainfont{Linux Libertine O}
\setsansfont{Linux Biolinum O}
\setmonofont[Scale=MatchLowercase]{Source Code Pro}
% \setmonofont{Inconsolata-g} % Skinny and easy on the eyes
% \setmathfont{TG Pagella Math}

% This is i18n overkill
% Also - folks seem to think polyglossia is better now (but support in LuaTeX is
% still a WIP in TeX Live 2013)
\usepackage{polyglossia}
\setdefaultlanguage[variant=american]{english}

% Disable morpheme-boundary-crossing ligatures
\usepackage[american]{selnolig}

% i18n of quotes - use for biblatex, but also potentially useful outside
% \usepackage{csquotes}

%% For bib if I end up using it.

% \usepackage[style=apa,backend=biber]{biblatex}

% More i18n overkill
% \DeclareLanguageMapping{american}{american-apa}

% \addbibresource{BIDS-app-sp2014.bib}

% Other setup:

\usepackage{underscore}
% Make it easier to print complicated tables
% \usepackage{longtable,siunitx,tabu}

% Better spacing and rules (lines, not laws)
% \usepackage{booktabs}

% \usepackage[small,compact]{titlesec}

% \usepackage{enumitem}
% \setlist{itemsep=0pt}

% \usepackage{minted}

% Note: with BibLaTeX, you should load hyperref after the biblatex package. And
% according to the WikiBook - it should be loaded close to last.
% pdfusetitle doesn't do much here (there's no TOC), but may as well keep it.
\usepackage[hidelinks,pdfusetitle]{hyperref}


% definitions/configuration for the header
\rehead[]{Dav Clark -- \textbf{Proposal} -- Data Science for Social Impact}
\lohead[]{Dav Clark -- \textbf{Proposal} -- Data Science for Social Impact}
\lehead[]{\pagemark}
\rohead[]{\pagemark}

\begin{document}
% that describes your efforts in data science, including your challenges and
% opportunities. No budget or a formal proposal.

% Instead, please provide an overview of the scientific goals and vision

The D-Lab and BIDS both have the same fundamental problem: \emph{how do we get
    collaboration right?} The topic of my PhD thesis was climate change, and
during the course of my study I became fully aware that we are in an era where
science could markedly decrease human suffering and species loss on a global
scale. Unfortunately, the practice of science is often heavily proscribed by
seemingly insurmountable technical challenges and barriers to collaboration. In
my experience, many “insurmountable challenges” are not even \emph{difficult} to
a relevant domain expert. In decision making terms, there is “money on the
table” in the form of the vast expertise embodied by researchers at UC Berkeley.
I propose to work with BIDS to identify low-hanging fruit for collaboration that
can push the frontiers of our ability to understand and engineer solutions for
our world. I am well suited to such an endeavor given my particular areas of
deep and broad expertise combined with my passion for impactful science.

\subsection*{Data Science for Social Impact}

If one had to pick “the” conference on data science, it would likely be the ACM KDD
(Knowledge Discovery and Data Mining) conference. The theme for KDD this year is
“Data Mining for Social Good.” 

\subsection*{So what \emph{can} I do?}

% Stuff from talking with Seb
% MGH, multimodal imaging
% Background in computational linguistics
% Crowdsourcing, fluency with latent variable models
% Engagement with real world problems
% DSSG
% Start-Up Chile

Before I came to Berkeley, I developed novel analysis technique using
inter-subject correlation in fMRI signals to observe the neural correlates of
ecological learning.\footnote{Specifically, viewing an episode of \emph{Curb
Your Enthusiasm}} I developed suitable algorithms for the taxonomic
categorization of web pages for online advertising. Then, I started working with
Rich Ivry on the neural basis of motor skill learning.

All of this contributed in some respect to my thesis research: evaluating a
number of climate education interventions using crowdsourcing on Amazon
Mechanical Turk. I implemented a solution in JavaScript to prevent Turkers from
enrolling in more than one of my experiments (a common problem). %cite?  
I also implemented a simple geo-IP based tool to determine whether individuals
were correctly reporting their location (two participants had in fact managed to
bypass Amazon restrictions on U.S.-only participants). Lastly, I implemented a
hybrid natural langauge processing solution / google web services solution to
check for individuals copying textual answers from the web instead of honestly
answering the question (turns out several people did this, even though there was
no consequence of answering poorly, and I told them I'd provide the correct
answer later).

If you've ever done this kind of work, it will come as no surprise that I also
dealt with numerous unicode issues.

With my team at Oroeco, I've set up an analytics environment using Vagrant such
that, even though he has no idea what he's actually doing, our CEO can spin up a
Linux VM, sync our live database to a local PostgreSQL server, and look at
up-to-the-minute visualizations in IPython notebooks that we share via GitHub.

% how working within a larger data science context will further that vision.

\subsection*{Generalizing collaboration in BIDS and the D-Lab}

I've been working at the D-Lab on campus-wide collaboration since last fall. If
you refer to the “impacts” document in my application, you'll see a list of
things that I'm already doing to share general solutions to improve sceintific
collaboration. In fact, I've fascilitated arguably the first large public
offering from BIDS in conjunction with Software Carpentry. My primary approach
so far, consistent with the D-Lab, has been educational. I'm working with Rachel
Slaybough \& Katy Huff in nuclear engineering and our own equity and inclusion
team in the D-Lab on planning improved technical education and assessment. I'm
also organizing a track on computing skills for reproducibility at the upcoming
BITSS/ICPSR summer institute.

\subsubsection*{Methods Modules}

Engineering proceeds in part by standardization of a level of complexity, so
that innovation can occur at a higher level. Currently methods education and
implementation are done in a relatively \emph{ad hoc} fashion. One approach
(described more fully in the “impacts” document) is to develop consistent
computing environments so that everyone on campus can be “on the same page” in
terms of sharing software and code. Philip Stark has generated some excitement
for this approach, in addition to individuals across campus on both the staff
and academic side.

A more interesting and ambitious project came out of my discussion with Rayid
Ghani and Lester Mackey about our “Data Science for Social Good” (DSSG) efforts.
In short, while DSSG was able to recruit technically talented graduate and
undergraduate students for their program last year, many of these students had
never been exposed to machine learning methods. Thus, a disproportionate amount
of time was spent navigating and “figuring out” specific data mining methods.
Our agreed-upon goal is to identify methods that have been broadly successful,
and provide an index that helps program participants (and ultimately, any
researcher) identify appropriate methods and readily apply them to their
question or problem domain. Meanwhile, Puneet Kishor (Science and Data Policy,
Creative Commons) has asked me to fascilitate collaboration with him and Peter
Murray-Rust (U of Cambridge) and Ross Mounce (U of Bath) on the development of a
text data mining curriculum.
% And similarly, Tom Griffiths is in dialogue with us
% at the D-Lab regarding his recently awarded grant for the creation of a Center
% for Data-Intensive Psychological Science calls for enabling researcher access to
% largely untapped resources, like image data on Flickr.
% The flip side of this demand is expressed in Xmobile -- where a certain level
% of capacity has been built up, but there's no clear sense of how to scale the
% program.
I hope these examples
suffice to illustrate the large need for well-organized, accessible materials
for implementing modern methods of data analysis. BIDS seems like the logical
place to center such efforts on campus.

\subsubsection*{Curated Data Sets}

Above, I mentioned the effort involved in accessing data from the CoolClimate
project. I'm working with the Info-Energy Nexus to help that group collect and
curate data API libraries and locally cached datasets. Here in the D-Lab, Jon
Stiles and I are particularly interested in a data collection project
surrounding the impacts of the Richmond Bay campus on the surrounding
neighborhoods (and working to make those impacts positive!). Jon provides his
extensive experience as a data archivist, and Puneet Keshor at Creative Commons
has offered to lend his resources in navigating legal/copyright issues.

Mention efforts with Twitter data, in particular Alexey, help to come from the
API group and Jey K. at the AMP lab.

\begin{enumerate}
    \item Organize free flow of data between both physical and behavioral
        research surrounding climate change.

        The UC Berkeley campus (+ LBNL \& friends) has an amazing wealth of such
        researchers, and they all whittle away in silos that render their work
        fairly impotent. An index of climate / ecology data from UC Berkeley
        provides a challenging project, and also seems like a good point of
        connection with Karthik (the engine behind rOpenSci -- which I was
        suggesting is the best first approximation I know of an “open data”
        index).

    \item Organize a dataset surrounding the Richmond Bay campus.

        This is a major initiative that will have huge impacts on the city of
        Richmond.  On the flipside, the city of Richmond is the largest city
        with a green mayor (a signal of strong liberalism), and there are
        existing examples of high school students working productively with
        municipal datasets.

    \item Coordinate with the Data Science for Social Good fellows program
        (\url{http://dssg.io}).

        I've been working on ramping up a bay area hub to coordinate with DSSG,
        picking up community partners with analysis problems they can't
        currently solve, and finding “data scientists” who can help at Berkeley
        and Stanford (and later, elsewhere). I've got a nascent site up with
        Lester Mackey \& Andreas Stuhlmüller at Stanford:
        http://statsforchange.org

    \item Enabling collaborative evaluation with D-Lab / Software Carpentry style
        technical training workshops.

        The D-Lab is both hosting SW Carpentry workshops, as well as
        coordinating with Rachel Slaybaugh (professor in Nuke Eng.) and Katy
        Huff on assessment with a focus on equity and inclusion. In parallel,
        we've been ramping up our own efforts in this domain. Open data would
        allow for more of a “citizen science” approach that would be coherent
        with the Software Carpentry approach to grass-roots instruction.

\end{enumerate}

% Our intent is two-fold: To build a community of data science, and help us
% steer and evolve the mission of BIDS; and to help attract funding from other
% sources, including new grants, support from private donors, and corporate
% sponsorships.

\end{document}
