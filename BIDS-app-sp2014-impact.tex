% How will the resources provided by BIDS be leveraged for greater impact in
% your research group?

\documentclass[
  fontsize=12,
  english,
  paper=letter,
  headsepstyle=on,
  footinclude=off]
  {scrartcl}
\usepackage{scrpage2}
% \usepackage[footinclude=off]{typearea}
% \usepackage[letterpaper,margin=1in,driver=pdftex]{geometry}

\clearscrheadfoot                 % deletes header/footer
\pagestyle{scrheadings}           % use following definitions for header/footer
% definitions/configuration for the header
\lehead[]{\pagemark} % equal page, left (outer) position
\rohead[]{\pagemark}

% This lets us load OpenType fonts from the system without much fuss.
% It's already required by unicode-math, but I'm not using math here (and it's
% worth being redundant)!
\usepackage{fontspec}

% What can I say, I love the french
\frenchspacing

% microtype enables subtle typography improvements.
\usepackage{microtype}
\DisableLigatures{family = tt*}

% This keeps things like `` and --- working, even though I use “ and ” now...
\defaultfontfeatures{Ligatures=TeX}

\setmainfont{Linux Libertine O}
\setsansfont{Linux Biolinum O}
\setmonofont{Source Code Pro}
% \setmonofont{Inconsolata-g} % Skinny and easy on the eyes
% \setmathfont{TG Pagella Math}

% This is i18n overkill
% Also - folks seem to think polyglossia is better now (but support in LuaTeX is
% still a WIP in TeX Live 2013)
\usepackage{polyglossia}
\setdefaultlanguage[variant=american]{english}

% Disable morpheme-boundary-crossing ligatures
\usepackage[american]{selnolig}

% i18n of quotes - use for biblatex, but also potentially useful outside
% \usepackage{csquotes}

%% For bib if I end up using it.

% uniquelist fixes a bug in mis-interpreting APA section 6.12, fixed in v. 6.1
% of the apa style
% \usepackage[style=apa,backend=biber]{biblatex}

% More i18n overkill
% \DeclareLanguageMapping{american}{american-apa}

% \addbibresource{BIDS-app-sp2014.bib}

% Other setup:

\usepackage{underscore}
% Make it easier to print complicated tables
% \usepackage{longtable,siunitx,tabu}

% Better spacing and rules (lines, not laws)
% \usepackage{booktabs}

% \usepackage[small,compact]{titlesec}

\usepackage{enumitem}
\setlist{itemsep=0pt}

% \usepackage{minted}

% Note: with BibLaTeX, you should load hyperref after the biblatex package. And
% according to the WikiBook - it should be loaded close to last.
% pdfusetitle doesn't do much here (there's no TOC), but may as well keep it.
\usepackage[hidelinks,pdfusetitle]{hyperref}

% even page, right position (inner)
\rehead[]{Dav Clark -- \textbf{Impact} -- Data Science for Social Impact}
% odd page, left position (inner)
\lohead[]{Dav Clark -- \textbf{Impact} -- Data Science for Social Impact}

\begin{document}

At the D-Lab, I interface with groups ranging from traditional social science
departments tothe Berkeley Institute for Transparency in the Social Sciences
(BITSS) to  UCSF/UCB purchasing. I'm “the data scientist” at the D-Lab --
an unwieldy proposition! To meet this challenge, I've networked with an enormous
variety of organizations on campus.  These contacts can range from working to
solve individual's immediate data analysis problems to thinking about how to
build campus-wide solutions that really scale. Here are examples of things I'm
already doing:

\begin{enumerate}
    \item Organizing the first publicly visible event for BIDS: the sold-out
        BIDS Software Carpentry Bootcamp
        (\url{http://swcarpentry.github.io/2014-03-17-ucb}, March 17-18).
    \item Participating in climate, energy, \& ecology oriented research groups
        on campus -- for example, teaching about git and life cycle analysis
        (LCA) at the Information-Energy Nexus.
    \item Providing internally- and externally-facing training both in the
        D-Lab, and also for sister organizations on campus, like the computing
        topics session at the BITSS/ICPSR summer institute:
        \url{http://bitss.org/2014/02/13/announcing-bitss-summer-institute}.
    \item Working with a variety of stakeholders on the development of
        high-quality curricula. I facilitate the D-Lab methods reading group
        (\url{http://dlab-berkeley.github.io/dlab-methods}), and have been
        working with educators inside and outside the D-Lab on planning
        curricula for traditionally under-served populations. Notably, I
        assisted Chris Holdgraf (graduate student in neuroscience) in writing
        his BIDS proposal to share a practical statistics curriculum.
    \item Together with some amazing support from the community, I've revived
        and now organize the scientific Python meetings on campus:
        \url{http://python.berkeley.edu/events}. (Fernando used to run this before he
        got too busy with larger things!)
    \item Pursuing Philip Stark's plan to provide a standardized campus computing
        environment for learning and research -- anywhere from your laptop to
        “the cloud.” Here, I'm collecting input from staff and faculty in EECS,
        Statistics, Shared Research Computing, the D-Lab, and the iSchool in
        addition to advice from members of the private sector data science community.
    \item Coordinating with Lester Mackey (professor of statistics at Stanford)
        on creating a Bay Area hub for socially engaged data science:
        \url{http://statsforchange.org}.
    \item Further coordinating this initiative with the broader Data Science for
        Social Good program at the Urban Data Center at U Chicago:
        \url{http://dssg.io}. This program is led by Matt Gee and Rayid Ghani, who were
        instrumental in setting the theme for this years KDD conference, “Data
        Mining for Social Good.”
    \item Providing infrastructure to make John Canny's GPU accelerated text
        parser available to social scientists.
    \item Developing no-cost solutions to organize lightweight groups on campus
        using GitHub.
\end{enumerate}

As a BIDS fellow, I will connect the BIDS community  with the groups I'm already
engaged with. If my experience is any judge, this will result in immense value
-- both in surfacing the really interesting, challenging problems, and sharing
effective, scalable solutions.

\end{document}
