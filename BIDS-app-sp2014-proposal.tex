\documentclass[
  fontsize=12,
  english,
  paper=letter,
  headsepstyle=on,
  footinclude=off]
  {scrartcl}
\usepackage{scrpage2}
% \usepackage[footinclude=off]{typearea}
% \usepackage[letterpaper,margin=1in,driver=pdftex]{geometry}

\clearscrheadfoot                 % deletes header/footer
\pagestyle{scrheadings}           % use following definitions for header/footer
% definitions/configuration for the header
\lehead[]{\pagemark} % equal page, left (outer) position
\rohead[]{\pagemark}

% This lets us load OpenType fonts from the system without much fuss.
% It's already required by unicode-math, but I'm not using math here (and it's
% worth being redundant)!
\usepackage{fontspec}

% What can I say, I love the french
\frenchspacing

% microtype enables subtle typography improvements.
\usepackage{microtype}
\DisableLigatures{family = tt*}

% This keeps things like `` and --- working, even though I use “ and ” now...
\defaultfontfeatures{Ligatures=TeX}

\setmainfont{Linux Libertine O}
\setsansfont{Linux Biolinum O}
\setmonofont{Source Code Pro}
% \setmonofont{Inconsolata-g} % Skinny and easy on the eyes
% \setmathfont{TG Pagella Math}

% This is i18n overkill
% Also - folks seem to think polyglossia is better now (but support in LuaTeX is
% still a WIP in TeX Live 2013)
\usepackage{polyglossia}
\setdefaultlanguage[variant=american]{english}

% Disable morpheme-boundary-crossing ligatures
\usepackage[american]{selnolig}

% i18n of quotes - use for biblatex, but also potentially useful outside
% \usepackage{csquotes}

%% For bib if I end up using it.

% uniquelist fixes a bug in mis-interpreting APA section 6.12, fixed in v. 6.1
% of the apa style
% \usepackage[style=apa,backend=biber]{biblatex}

% More i18n overkill
% \DeclareLanguageMapping{american}{american-apa}

% \addbibresource{BIDS-app-sp2014.bib}

% Other setup:

\usepackage{underscore}
% Make it easier to print complicated tables
% \usepackage{longtable,siunitx,tabu}

% Better spacing and rules (lines, not laws)
% \usepackage{booktabs}

% \usepackage[small,compact]{titlesec}

\usepackage{enumitem}
\setlist{itemsep=0pt}

% \usepackage{minted}

% Note: with BibLaTeX, you should load hyperref after the biblatex package. And
% according to the WikiBook - it should be loaded close to last.
% pdfusetitle doesn't do much here (there's no TOC), but may as well keep it.
\usepackage[hidelinks,pdfusetitle]{hyperref}


% definitions/configuration for the header
\rehead[]{Dav Clark -- \textbf{Proposal} -- Data Science for Social Impact}
\lohead[]{Dav Clark -- \textbf{Proposal} -- Data Science for Social Impact}
\lehead[]{\pagemark}
\rohead[]{\pagemark}
% Describe the intended research project in more detail, including the
% anticipated impact of your proposed project on advancing scientific discovery
% in your field. (3 page maximum)

% I had been working on these as a separate application:
% 
%  - Could also involve Jon & Richmond Bay, or more generally Bay Area hub (Richmond + Oakland, etc)
%  - Proposal to BIDS as instructional development grant on picking up ill-fitting
%    project from DSSG.io and existing local
%  - $20k as a starting budget for a GSR to work on this
%  - Anno could support a GSR?
%  - Mention Bay Area hub, but call out Richmond Bay.
% Play up KDD Data Mining for Social Good

\begin{document}

% \title{Data Science for Social Impact}
% \author{Dav Clark} % must match BearFacts!
% \maketitle
\manualmark
\pagestyle{scrheadings}
\markleft{}

% Deep - Climate Change, Human Learning \& Behavior Change, Hacking skills

% Broad - Knowledge of campus, exposure to a variety of fields over the course
% of my training and now in the D-Lab.

% What is the "great project"?

Below, I'll discuss how I've personally encoutered the importance of
collaboration in my own research and describe my current efforts to surmount
challenges to effective climate-related behavior change. In persuing this
program of research, I will continuously work to develop general practices that
will help BIDS and the D-Lab develop effective collaborations, both internally
and externally.

\subsection*{Climate change education \& behavior change}
% Maybe mention the flip w/ liberals and nuclear power?

% Climate!
% Oroeco
% Campus partners - Kevin / GIF, Karthik?, Zeke / BEST, Ranney, etc., Clayton Critcher?
% Identified problems with physical / behavioral systems (cf. Sol, Ted Miguel)
% Quote via Steve Lewendowski (IPCC guy - problem is behavioral, not physical), more likely Elke Weber
% Garrison, BECC, etc.

Global warming akin to recent and projected trends last occurred over 17 million
years ago, when a 3-4°C gain occurred over 1,500,000 years. The current trend
will likely yeild comparable warming in 100 years. Historically, warming is
correlated with extinction, so this is really bad. This challenge presents a
tremendous opportunity for science to be employed across disciplinary boundaries
-- ranging from climatology to behavior change to public policy -- and in the
process prevent mass extinctions and human suffering. \emph{But}, we don't yet
have the capacity to do this kind of science. I would like to collaborate with
BIDS on enabling the capacity to do meaningful collaborative work in the context
of my research on climate systems and behavior. In the D-Lab, I'm developing
tools and helping social scientists gain skills that enable access to new
sources of data and new methods of analysis.  Working with a network of peers, I
can rapidly iterate on approaches that will help address climate change and
other global challenges.

The urgency and importance of meeting the challenge of scientific collaboration
became particularly apparent to me in my thesis work on climate change education.
There, models of increasing technical and scientific sophistication are
continually increasing our certainty and understanding of the very real dangers
facing our world. In spite of the clarity and urgency revealed by the physical
sciences, meaningful policy change has been extraordinarily difficult to bring
about. Indeed, only in the most recent IPCC report (the fifth so far) was a
social scientist (Elke Weber) invited to contribute to recommendations for
mitigating the likely dire consequences of global climate change. Sadly, the
predominant opinion is that “education” is not helpful to combat climate change.
This is, I argue in my thesis, partly due to the way academic pressures lead to
structuring these questions. When one takes the time to be thoughtful and
pragmatic, it appears that “education” does \emph{not} necessarily have
polarizing effects. Rather, it can yeild small but significant gains in climate
change acceptance.

% http://theconversation.com/the-truth-is-out-there-so-how-do-you-debunk-a-myth-22641?utm_source=feedburner&utm_medium=feed&utm_campaign=Feed%3A+conversationedu+(The+Conversation)
% John Cook at Skeptical Science and the University of Queensland might be
% considered a data science pioneer in the climate and behavior field. He runs
% what is likely the highest profile “climate truth” website, and had a successful
% campaign largely via Twitter and Facebook last year to fight misinformation
% about scientific consensus with this verified fact: over 97\% of publishing
% climate researchers accept the reality of human-caused climate change.

In my “spare time,”\footnote{This nominally includes up to 10\% of my time in
    the D-Lab, though I am usually too busy with core D-Lab business.} I now
serve as chief scientist at Oroeco, a not-for-profit start-up that takes a
consumer-spending approach to reducing carbon footrint.  Our CEO has worked
extensively as a life-cycle analysis consultant and lecturer in the Precourt
center at Stanford. We're applying that expertise to identify the specific
changes an individual can make\footnote{We can base recommendations on
    geographical location and individual purchasing behavior.} and provide
behavior change scaffolds to help individuals focus on high-impact goals. Our
general approach has been tried at least twice before (at WattzOn and Efficiency
2.0) and those businesses have failed.\footnote{Technically, they pivoted. A
    quick search will reveal that both brands are still around.} However, we
have MacArthur “genius” Saul Griffith -- a founder of WattzOn -- as an advisor
and he believes it's worth another shot, especially given advances in working
through social media and mobile interactivity. Likewise, Zeke Hausfather -- a
founding member of Efficiency 2.0 who is now working with Richard Mueller at the
Berkeley Earth Institute -- is working with us on implementing effective tips
and goals for our users. Two governmental start-up
accelerators\footnote{Start-Up Chile and the DOD/DOE Energy Accelerator} have
provided us with close to \$100,000 in equity-free funding, which is perhaps the
strongest endorsement. I hope \emph{you} agree that it's worth another try.

The curent reality is that almost no one in the climate and behavior field has
the “hacking skills” needed to do truly first-rate data science in the field of
climate and behavior. To a behavioral scientist the research done by WattzOn and
Efficiency 2.0 (and even Oroeco, for now) is pretty simple-minded. Climate
scientists certainly have access to incredible hacking skills for solving
massive parallel systems of dynamic equations, but they don't have the
combination of social-behavioral science and facility with web technologies and
social media required to do a the job we need to be doing.

On campus, there are numerous individuals and groups who are supportive of this
project.  The most notable is Chris Jones, who works with Dan Kammen on the
CoolCalifornia project.  This project provides a lot of the information we need
to help Oroeco's users figure out the big impacts they can make in their
individual spending patterns.  Simply getting access to this data took far
longer than it might have, in part because Chris and his team are busy dealing
with traditional academic pressures. Recently, CoolClimate partnered with the
campus API group, which greatly simplified access for groups like
Oroeco.\footnote{See \url{http://developer.berkeley.edu} for more information on
    this excellent program.} This engagement with the API group is already a
success case that I'd like to see replicated across other useful campus data
sources. But even with CoolClimate, a lot of unrealized potential remains.

Now that the foundations are in place, I am leveraging my participation in the
Behavioral Change \& Measurement group on campus.\footnote{A Social Science
    Matrix seminar, see \url{http://xmobile.berkeley.edu/seminar/} for details.}
I've already presented the basis of our approach at Oroeco there, and I've
gotten rich feedback from a group that spans engineers, policy researchers, and
psychologists. One member of this group, Pablo Paredes, is providing targeted
consulting based on his experience with behavior change and gamification at
Microsoft. The short answer is that the process of developing a good
intervention is \emph{hard}, and it requires social science methods to develop
understanding of the kinds of users you have.

% Mention collaboration with game design thesis at UC Irvine

% Go on to campus efforts / Energy Information Nexus

\subsubsection{Specific Aims}

Species are going extinct, % cite?
and humans are already suffering as a
consequence of global climate change. %Hsiang
This will continue to happen, but
it is within our power to make it a little more or a little less. By myself, I
don't have the skills or bandwidth to properly design, evaluate, and rapidly
iterate on the conceptual and behavioral change interventions that would
mitigate climate change and its impacts. I certainly don't have the documented
billion dollars that have been spent on anti-science misinformation.

\end{document}
