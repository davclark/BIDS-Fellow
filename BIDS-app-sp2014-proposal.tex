\documentclass[
  fontsize=12,
  english,
  paper=letter,
  headsepstyle=on,
  footinclude=off]
  {scrartcl}
\usepackage{scrpage2}
% \usepackage[footinclude=off]{typearea}
% \usepackage[letterpaper,margin=1in,driver=pdftex]{geometry}

\clearscrheadfoot                 % deletes header/footer
\pagestyle{scrheadings}           % use following definitions for header/footer
% definitions/configuration for the header
\lehead[]{\pagemark} % equal page, left (outer) position
\rohead[]{\pagemark}

% This lets us load OpenType fonts from the system without much fuss.
% It's already required by unicode-math, but I'm not using math here (and it's
% worth being redundant)!
\usepackage{fontspec}

% What can I say, I love the french
\frenchspacing

% microtype enables subtle typography improvements.
\usepackage{microtype}
\DisableLigatures{family = tt*}

% This keeps things like `` and --- working, even though I use “ and ” now...
\defaultfontfeatures{Ligatures=TeX}

\setmainfont{Linux Libertine O}
\setsansfont{Linux Biolinum O}
\setmonofont{Source Code Pro}
% \setmonofont{Inconsolata-g} % Skinny and easy on the eyes
% \setmathfont{TG Pagella Math}

% This is i18n overkill
% Also - folks seem to think polyglossia is better now (but support in LuaTeX is
% still a WIP in TeX Live 2013)
\usepackage{polyglossia}
\setdefaultlanguage[variant=american]{english}

% Disable morpheme-boundary-crossing ligatures
\usepackage[american]{selnolig}

% i18n of quotes - use for biblatex, but also potentially useful outside
% \usepackage{csquotes}

%% For bib if I end up using it.

% uniquelist fixes a bug in mis-interpreting APA section 6.12, fixed in v. 6.1
% of the apa style
% \usepackage[style=apa,backend=biber]{biblatex}

% More i18n overkill
% \DeclareLanguageMapping{american}{american-apa}

% \addbibresource{BIDS-app-sp2014.bib}

% Other setup:

\usepackage{underscore}
% Make it easier to print complicated tables
% \usepackage{longtable,siunitx,tabu}

% Better spacing and rules (lines, not laws)
% \usepackage{booktabs}

% \usepackage[small,compact]{titlesec}

\usepackage{enumitem}
\setlist{itemsep=0pt}

% \usepackage{minted}

% Note: with BibLaTeX, you should load hyperref after the biblatex package. And
% according to the WikiBook - it should be loaded close to last.
% pdfusetitle doesn't do much here (there's no TOC), but may as well keep it.
\usepackage[hidelinks,pdfusetitle]{hyperref}


% definitions/configuration for the header
\rehead[]{Dav Clark -- \textbf{Proposal} -- Data Science for Social Impact}
\lohead[]{Dav Clark -- \textbf{Proposal} -- Data Science for Social Impact}
\lehead[]{\pagemark}
\rohead[]{\pagemark}
% Describe the intended research project in more detail, including the
% anticipated impact of your proposed project on advancing scientific discovery
% in your field. (3 page maximum)

% I had been working on these as a separate application:
% 
%  - Could also involve Jon & Richmond Bay, or more generally Bay Area hub (Richmond + Oakland, etc)
%  - Proposal to BIDS as instructional development grant on picking up ill-fitting
%    project from DSSG.io and existing local
%  - $20k as a starting budget for a GSR to work on this
%  - Anno could support a GSR?
%  - Mention Bay Area hub, but call out Richmond Bay.
% Play up KDD Data Mining for Social Good

\begin{document}

% \title{Data Science for Social Impact}
% \author{Dav Clark} % must match BearFacts!
% \maketitle
\manualmark
\pagestyle{scrheadings}
\markleft{}

The D-Lab and BIDS both have the same fundamental problem: \emph{how do we get
collaboration right?} My thesis, for example, examined pervasive misconceptions
about climate change education. We are in an era where science could markedly
decrease human suffering and species loss on a global scale. Unfortunately, the
practice of science is heavily proscribed by seemingly insurmountable technical
challenges and barriers to collaboration. In my experience, many “insurmountable
challenges” are not even \emph{difficult} to a relevant domain expert. In
decision making terms, there is “money on the table” in the form of the vast
expertise embodied by researchers at UC Berkeley. I propose to work with BIDS to
identify low-hanging fruit for collaboration that can push the frontiers of our
ability to understand and engineer solutions for our world. I am well suited to
such an endeavor given my particular areas of deep and broad expertise combined with my
passion for impactful science.

% Deep - Climate Change, Human Learning \& Behavior Change, Hacking skills

% Broad - Knowledge of campus, exposure to a variety of fields over the course
% of my training and now in the D-Lab.

% What is the "great project"?

Below, I'll discuss how I've personally encoutered the importance of
collaboration in my own research and describe my current efforts to surmount
challenges to effective climate-related behavior change. From there, I'll
proceed to describe how I'd like to generalize these practices to help BIDS
develop effective collaborations, both internally and externally.

\subsection*{A case study: Climate change education \& behavior change}
% Maybe mention the flip w/ liberals and nuclear power?

% Climate!
% Oroeco
% Campus partners - Kevin / GIF, Karthik?, Zeke / BEST, Ranney, etc., Clayton Critcher?
% Identified problems with physical / behavioral systems (cf. Sol, Ted Miguel)
% Quote via Steve Lewendowski (IPCC guy - problem is behavioral, not physical), more likely Elke Weber
% Garrison, BECC, etc.

The urgency and importance of meeting the challenge of scientific collaboration
became particularly apparent in my thesis work on climate change education.
There, models of increasing technical and scientific sophistication are
continually increasing our certainty and understanding of the very real dangers
facing our world. In spite of the clarity and urgency revealed by the physical
sciences, meaningful policy change has been extraordinarily difficult to bring
about. Indeed, only in the most recent IPCC report (the fifth so far) was a
social scientist (Elke Weber) invited to contribute to recommendations for
mitigating the likely dire consequences of global climate change. Sadly, the
predominant opinion is that “education” is not helpful to combat climate change.
This is, I argue in my thesis, partly due to the way academic pressures lead to
structuring these questions. When one takes the time to be thoughtful and
pragmatic, it appears that “education” does \emph{not} necessarily have
polarizing effects, but rather can yeild small but significant gains in climate
change acceptance.

% http://theconversation.com/the-truth-is-out-there-so-how-do-you-debunk-a-myth-22641?utm_source=feedburner&utm_medium=feed&utm_campaign=Feed%3A+conversationedu+(The+Conversation)
% John Cook at Skeptical Science and the University of Queensland might be
% considered a data science pioneer in the climate and behavior field. He runs
% what is likely the highest profile “climate truth” website, and had a successful
% campaign largely via Twitter and Facebook last year to fight misinformation
% about scientific consensus with this verified fact: over 97\% of publishing
% climate researchers accept the reality of human-caused climate change.

In my “spare time,”\footnote{This nominally includes up to 10\% of my time in
    the D-Lab, though I am usually too busy with core D-Lab business.} I serve as
chief scientist at Oroeco, a not-for-profit start-up that takes a
consumer-spending approach to reducing carbon footrint.  Our CEO has worked
extensively as a life-cycle analysis consultant and lecturer in the Precourt
center at Stanford. We're applying that expertise to identify the specific
changes an individual can make\footnote{We can base recommendations on
    geographical location and individual purchasing behavior.} and provide
behavior change scaffolds to help individuals focus on high-impact goals. Our
general approach has been tried at least twice before (at WattzOn and Efficiency
2.0) and those businesses have failed.\footnote{Technically, they pivoted. A
    quick search will reveal that both brands are still around.} However, we
have MacArthur “genius” Saul Griffith -- a founder of WattzOn -- as an advisor
and he believes it's worth another shot, especially given advances in working
through social media and mobile interactivity. Likewise, Zeke Hausfather -- a
founding member of Efficiency 2.0 who is now working with Richard Mueller at the
Berkeley Earth Institute -- is working with us on implementing effective tips
and goals for our users. Two governmental start-up
accelerators\footnote{Start-Up Chile and the DOD/DOE Energy Accelerator} have
provided us with close to \$100,000 in equity-free funding, which is perhaps the
strongest endorsement. I hope \emph{you} agree that it's worth another try.

The curent reality is that almost no one in the climate and behavior field has
the “hacking skills” needed to do truly first-rate data science in the field of
climate and behavior. To a behavioral scientist the research done by WattzOn and
Efficiency 2.0 (and even Oroeco, for now) is pretty simple-minded. Climate
scientists certainly have access to incredible hacking skills for solving
massive parallel systems of dynamic equations, but they don't have the
combination of social-behavioral science and facility with web technologies and
social media required to do a the job we need to be doing.

On campus, there are numerous individuals who are supportive of this project.
The most notable is likely Chris Jones and Dan Kammen's CoolCalifornia project.
This project provides a lot of the information we need to help Oroeco's users
figure out the big impacts they can make in their individual spending patterns.
Simply getting access to this data took far longer than it might have, Chris is
busy dealing with traditional academic pressures. Recently, CoolClimate
partnered with the campus API group, which greatly simplified access for groups
like Oroeco.\footnote{See
    \url{http://developer.berkeley.edu} for more information on this excellent
    program.} While initiatives like This is already a success case, but a
lot of unrealized potential remains.

Now that the foundations are in place, I am leveraging my participation in the
Behavioral Change \& Measurement group on campus.\footnote{A Social Science
    Matrix seminar, see \url{http://xmobile.berkeley.edu/seminar/} for details.}
I've already presented the basis of our approach at Oroeco there, and I've
gotten rich feedback from a group that spans engineers, policy researchers, and
psychologists. One member of this group, Pablo Paredes, is providing targeted
consulting based on his experience with behavior change and gamification at
Microsoft. The short answer is that the process of developing a good
intervention is \emph{hard}, and it requires social science methods to develop
understanding of the kinds of users you have.
% Mention collaboration with game design thesis at UC Irvine

% Go on to campus efforts / Energy Information Nexus

Species are going extinct, % cite?  and humans are already suffering as a
consequence of global climate change. %Hsiang This will continue to happen, but
it is within our power to make it a little more or a little less. By myself, I
don't have the skills or bandwidth to properly design, evaluate, and rapidly
iterate on the conceptual and behavioral change interventions that would
mitigate climate change and its impacts. I certainly don't have the documented
billion dollars that have been spent on anti-science misinformation.

We need to get our act together on scientific collaboration, but established
academic pressures keep people like Chris Jones busy working in a relatively
low-impact way.\footnote{Compared to the norms in academia, Chris is a
    model of high-impact! But he could get a lot more done with the appropriate
    structures and resources.} Happily, developments with the API group have
improved things quite a bit, but Chris is still working with the model of hiring
a single undergraduate to continue work on the project when he has the time and
resources to support development. We are only now -- \emph{after several years
    of working hard to get things in place} -- able to begin more
sophisticated evaluations of real behavior with the Oroeco platform.

There's still plenty of work to do to support climate and behavior research with
the technologies and approaches I'm currently developing. Beyond Oroeco, I'm
working with a Beth Karlin and David Conley at UC Irvine on their efforts to
evaluate emotive artwork surrounding climate change. I'm also working locally
with the Energy-Information Nexus working group to provide the skills and
impetus for folks to collaborate using free (or low-cost) approaches like
GitHub.

It's clear that integrating knowledge -- both via sound data management and a
collaborative culture of scientific practice --  can be incredibly high-profile,
as with Solomon Hsiang's excellent work on coupled physical-behavioral systems.
I think the work I'm doing on climate and behavior stands to have similar
impact.  I'd love to more formally “collaborate on collaboration” with you at
BIDS -- putting the right structures and resources in place in a systematic way
both for advancing the practice and impact of science both on and off campus.

\subsection*{Generalizing collaboration in BIDS and the D-Lab}

I've been working at the D-Lab on campus-wide collaboration since last fall. If
you refer to the “impacts” document in my application, you'll see a list of
things that I'm already doing to share general solutions to improve sceintific
collaboration. In fact, I've fascilitated arguably the first large public
offering from BIDS in conjunction with Software Carpentry. My primary approach
so far, consistent with the D-Lab, has been educational. I'm working with Rachel
Slaybough \& Katy Huff in nuclear engineering and our own equity and inclusion
team in the D-Lab on planning improved technical education and assessment. I'm
also organizing a track on computing skills for reproducibility at the upcoming
BITSS/ICPSR summer institute.

\subsubsection*{Methods Modules}

Engineering proceeds in part by standardization of a level of complexity, so
that innovation can occur at a higher level. Currently methods education and
implementation are done in a relatively \emph{ad hoc} fashion. One approach
(described more fully in the “impacts” document) is to develop consistent
computing environments so that everyone on campus can be “on the same page” in
terms of sharing software and code. Philip Stark has generated some excitement
for this approach, in addition to individuals across campus on both the staff
and academic side.

A more interesting and ambitious project came out of my discussion with Rayid
Ghani and Lester Mackey about our “Data Science for Social Good” (DSSG) efforts.
In short, while DSSG was able to recruit technically talented graduate and
undergraduate students for their program last year, many of these students had
never been exposed to machine learning methods. Thus, a disproportionate amount
of time was spent navigating and “figuring out” specific data mining methods.
Our agreed-upon goal is to identify methods that have been broadly successful,
and provide an index that helps program participants (and ultimately, any
researcher) identify appropriate methods and readily apply them to their
question or problem domain. Meanwhile, Puneet Kishor (Science and Data Policy,
Creative Commons) has asked me to fascilitate collaboration with him and Peter
Murray-Rust (U of Cambridge) and Ross Mounce (U of Bath) on the development of a
text data mining curriculum.
% And similarly, Tom Griffiths is in dialogue with us
% at the D-Lab regarding his recently awarded grant for the creation of a Center
% for Data-Intensive Psychological Science calls for enabling researcher access to
% largely untapped resources, like image data on Flickr.
% The flip side of this demand is expressed in Xmobile -- where a certain level
% of capacity has been built up, but there's no clear sense of how to scale the
% program.
I hope these examples
suffice to illustrate the large need for well-organized, accessible materials
for implementing modern methods of data analysis. BIDS seems like the logical
place to center such efforts on campus.

\subsubsection*{Curated Data Sets}

Above, I mentioned the effort involved in accessing data from the CoolClimate
project. I'm working with the Info-Energy Nexus to help that group collect and
curate data API libraries and locally cached datasets. Here in the D-Lab, Jon
Stiles and I are particularly interested in a data collection project
surrounding the impacts of the Richmond Bay campus on the surrounding
neighborhoods (and working to make those impacts positive!). Jon provides his
extensive experience as a data archivist, and Puneet Keshor at Creative Commons
has offered to lend his resources in navigating legal/copyright issues.

Mention efforts with Twitter data, in particular Alexey, help to come from the
API group and Jey K. at the AMP lab.

\begin{enumerate}
    \item Organize free flow of data between both physical and behavioral
        research surrounding climate change.

        The UC Berkeley campus (+ LBNL \& friends) has an amazing wealth of such
        researchers, and they all whittle away in silos that render their work
        fairly impotent. An index of climate / ecology data from UC Berkeley
        provides a challenging project, and also seems like a good point of
        connection with Karthik (the engine behind rOpenSci -- which I was
        suggesting is the best first approximation I know of an “open data”
        index).

    \item Organize a dataset surrounding the Richmond Bay campus.

        This is a major initiative that will have huge impacts on the city of
        Richmond.  On the flipside, the city of Richmond is the largest city
        with a green mayor (a signal of strong liberalism), and there are
        existing examples of high school students working productively with
        municipal datasets.

    \item Coordinate with the Data Science for Social Good fellows program
        (\url{http://dssg.io}).

        I've been working on ramping up a bay area hub to coordinate with DSSG,
        picking up community partners with analysis problems they can't
        currently solve, and finding “data scientists” who can help at Berkeley
        and Stanford (and later, elsewhere). I've got a nascent site up with
        Lester Mackey \& Andreas Stuhlmüller at Stanford:
        http://statsforchange.org

    \item Enabling collaborative evaluation with D-Lab / Software Carpentry style
        technical training workshops.

        The D-Lab is both hosting SW Carpentry workshops, as well as
        coordinating with Rachel Slaybaugh (professor in Nuke Eng.) and Katy
        Huff on assessment with a focus on equity and inclusion. In parallel,
        we've been ramping up our own efforts in this domain. Open data would
        allow for more of a “citizen science” approach that would be coherent
        with the Software Carpentry approach to grass-roots instruction.

\end{enumerate}
\subsection*{So what \emph{can} I do?}

% Stuff from talking with Seb
% MGH, multimodal imaging
% Background in computational linguistics
% Crowdsourcing, fluency with latent variable models
% Engagement with real world problems
% DSSG
% Start-Up Chile

Before I came to Berkeley, I developed novel analysis technique using
inter-subject correlation in fMRI signals to observe the neural correlates of
ecological learning.\footnote{Specifically, viewing an episode of \emph{Curb
Your Enthusiasm}} I developed suitable algorithms for the taxonomic
categorization of web pages for online advertising. Then, I started working with
Rich Ivry on the neural basis of motor skill learning.

All of this contributed in some respect to my thesis research: evaluating a
number of climate education interventions using crowdsourcing on Amazon
Mechanical Turk. I implemented a solution in JavaScript to prevent Turkers from
enrolling in more than one of my experiments (a common problem). %cite?  
I also implemented a simple geo-IP based tool to determine whether individuals
were correctly reporting their location (two participants had in fact managed to
bypass Amazon restrictions on U.S.-only participants). Lastly, I implemented a
hybrid natural langauge processing solution / google web services solution to
check for individuals copying textual answers from the web instead of honestly
answering the question (turns out several people did this, even though there was
no consequence of answering poorly, and I told them I'd provide the correct
answer later).

If you've ever done this kind of work, it will come as no surprise that I also
dealt with numerous unicode issues.

With my team at Oroeco, I've set up an analytics environment using Vagrant such
that, even though he has no idea what he's actually doing, our CEO can spin up a
Linux VM, sync our live database to a local PostgreSQL server, and look at
up-to-the-minute visualizations in IPython notebooks that we share via GitHub.


\end{document}
